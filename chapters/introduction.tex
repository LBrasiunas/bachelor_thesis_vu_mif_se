\sectionnonum{Introduction}

The industrial revolution, which began in the late 18th century and continued into the 19th, had a profound impact on population growth, laying the foundation for the demographic explosion that the world knows today. It brought some important industrial changes including advancements in agriculture, public health, transportation, communication that dramatically altered living conditions without which humans would not be as successful as they are today. This major humanity milestone has advanced humans but as a consequence introduced some important problems that are being faced right now.

Today 4.4 billion people live in cities which is expected to increase, with the urban population more than doubling its current size by 2050, which would translate in 7 out of 10 people living in the cities \cite{WrldBnkUrbanDevelopment}. This shows the importance of city planning and world resource allocation and budgeting in order for humanity to be prosperous today and in the future. This means that country governments and other institutions need to conduct population forecasting censuses in order to make important decisions when it comes to existing and new city planning. Current methods of taking population censuses are outdated, take a large amount of time and are conducted quite rarely - at least once every 10 years in developed countries. Traditional current census conducting methods include registration of all individuals and their details using questionnaires during online and real-life campaigns which last from a few days to couple of weeks or even months. Another popular approach is conducted using population registers and other administrative sources without the need of collecting population information via questionnaires \cite{CensusApproachesAndMethods}. This shows that a need of new, faster, cheaper and more accurate, population forecasting methods analysis and implementation is essential.

This work includes detailed analysis of literature in order to understand current population forecasting methods in depth, their pitfalls and strengths, when it comes to a rapidly changing world and it's urban environment. The thesis describes both currently used population census conducting methods in detail and analyses an alternative method using data that is already widely available on the internet to analyse its' potential when it comes to population forecasting - using deep learning and neural network models.

This thesis analyses and compares the speed, accuracy and resources needed for a couple of chosen neural network models (\textbf{NAME THE CHOSEN MODELS HERE}) when it comes to forecasting population using satellite images. The analysis and comparison will be done using a custom dataset consisting of coloured orthographic maps and population statistics in a specific area.

This work will grant a conclusion as to which neural network model is best fit for this type of population forecasting using satellite images. The study aims to identify which model is best fit when it comes to population census based on different evaluation criteria such as speed, accuracy, loss and resources needed to train the model. To achieve the desired goal, tasks defined further need to be completed:
\begin{enumerate}
    \item Task 1
    \item Task 2
    \item Task 3
    \item Task 4
    \item Task 5
\end{enumerate}