\subsection{Population registry analysis method overview}

Population registry analysis is a method used to estimate and monitor population counts by leveraging administrative records rather than conducting traditional door-to-door censuses. These registries typically contain detailed information about individuals, such as birth and death records, migration data, and changes in residency. Using population registries for census purposes offers several advantages. First, it provides up-to-date and continuous information, enabling real-time monitoring of population changes. This is particularly useful for tracking migration patterns, urbanization, and demographic shifts. Second, it reduces the cost and logistical challenges associated with traditional censuses, as it eliminates the need for large-scale surveys or interviews \cite{RegisterBasedCensusStatistics}.

A key example is the population registry system used in the Nordic countries, such as Sweden, Norway, and Denmark. These nations maintain central population registers, where administrative data is continuously updated, making it easier to track population dynamics without the need for costly surveys \textbf{REF}.

\textbf{TO-DO: Continue writing about how the population registry analysis works under the hood (what is needed in order for it to be successful and efficient).}