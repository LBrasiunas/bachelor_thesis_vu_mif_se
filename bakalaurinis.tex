\documentclass{VUMIFPSbakalaurinis}
\usepackage{float}
\usepackage{wrapfig2}
\usepackage{hyperref}
\usepackage{algorithmicx}
\usepackage{algorithm}
\usepackage{algpseudocode}
\usepackage{amsfonts}
\usepackage{amsmath}
\usepackage{bm}
\usepackage{caption}
\usepackage{color}
\usepackage{graphicx}
\usepackage{listings}
\usepackage{subcaption}
\usepackage{biblatex}

% Titulinio aprašas
\university{Vilniaus universitetas}
\faculty{Matematikos ir informatikos fakultetas}
\institute{Informatikos institutas}  % Užkomentavus šią eilutę - institutas neįtraukiamas į titulinį
\department{Programų sistemų bakalauro studijų programa}
\papertype{Bakalauro baigiamasis darbas}
\title{Programų sistemų kūrimo metodų tyrimas}
\titleineng{Investigation of Methods of Software Development}
\author{Vardenis Pavardenis}
% \secondauthor{Vardonis Pavardonis}   % Pridėti antrą autorių
\supervisor{prof. habil. dr. Vardaitis Pavardaitis}
\reviewer{doc. dr. Vardauskas Pavardauskas}
\addsignatureplaces{} % prideda parašų vietas tituliniame puslapyje
\date{Vilnius – \the\year}

\bibliography{bibliografija}

\begin{document}
\maketitle

%% Padėkų skyrius
% \sectionnonumnocontent{}
% \vspace{7cm}
% \begin{center}
%     Padėkos asmenims ir/ar organizacijoms
% \end{center}

\sectionnonumnocontent{Santrauka}
Glaustai aprašomas darbo turinys: pristatoma nagrinėta problema ir padarytos
išvados. Santraukos apimtis ne didesnė nei 0,5 puslapio. Santraukų gale
nurodomi darbo raktiniai žodžiai.
% Nurodomi iki 5 svarbiausių temos raktinių žodžių (terminų).
% Vienas terminas gali susidėti iš kelių žodžių.
\raktiniaizodziai{raktinis žodis 1, raktinis žodis 2, raktinis žodis 3, raktinis žodis 4, raktinis žodis 5}

\sectionnonumnocontent{Summary}
Santrauka anglų kalba. Santraukos apimtis ne didesnė nei 0,5 puslapio.
\keywords{keyword 1, keyword 2, keyword 3, keyword 4, keyword 5}

\tableofcontents

\sectionnonum{Įvadas}
Įvade nurodomas darbo tikslas ir uždaviniai, kuriais bus įgyvendinamas tikslas,
aprašomas temos aktualumas, apibrėžiamas tiriamasis objektas akcentuojant
neapibrėžtumą, kuris bus išspręstas darbe, aptariamos teorinės darbo prielaidos
bei metodika, apibūdinami su tema susiję literatūros ar kitokie šaltiniai,
temos analizės tvarka, darbo atlikimo aplinkybės, pateikiama žinių apie
naudojamus instrumentus (programas ir kt., jei darbe yra eksperimentinė dalis).
Darbo įvadas neturi būti dėstymo santrauka. Įvado apimtis 2--4 puslapiai.

\section{Medžiagos darbo tema dėstymo skyriai}
Medžiagos darbo tema dėstymo skyriuose išsamiai pateikiamos nagrinėjamos temos
detalės: pradiniai duomenys, jų analizės ir apdorojimo metodai, sprendimų
įgyvendinimas, gautų rezultatų apibendrinimas.

Medžiaga turi būti dėstoma aiškiai, pateikiant argumentus. Tekste dėstomas
trečiuoju asmeniu, t.y. rašoma ne \enquote{aš manau}, bet „autorius mano“, „autoriaus
nuomone“. Reikėtų vengti informacijos nesuteikiančių frazių, pvz., „...kaip jau
buvo minėta...“, „...kaip visiems žinoma...“ ir pan., vengti grožinės
literatūros ar publicistinio stiliaus, gausių metaforų ar panašių meninės
išraiškos priemonių.

Skyriai gali turėti poskyrius ir smulkesnes sudėtines dalis, kaip punktus ir
papunkčius.

\subsection{Poskyris}
Citavimo pavyzdžiai: cituojamas vienas šaltinis \cite{PvzStraipsnLt}; cituojami
keli šaltiniai \cite{PvzStraipsnEn, PvzStraipsnLta, PvzKonfLt, PvzKonfEn, PvzKnygLt, PvzKnygEn,
PvzElPubLt, PvzElPubEn, PvzBakLt, PvzMagistrLt, PvzPhdEn}.

Anglų kalbos terminų pateikimo pavyzdžiai: priklausomybių injekcija (\anglnb{dependency injection},
dažnai trumpinama kaip \textit{DI}), saitų redaktorius \angl{linker}.

\subsection{Faktorialo algoritmas}

\ref{alg:factorial} pav. pateiktas faktorialo algoritmas.

\begin{algorithm}
\begin{algorithmic}[1] % [1] padaro, kad eilutės būtų sunumeruotos
\State $N\gets$ skaičius, kurio faktorialą skaičiuojame
\State $F\gets 1$
\For{$i := 2$ $to$ $N$}
    \State $F\gets F \cdot i$
\EndFor
\end{algorithmic}
\caption{Faktorialo algoritmas}
\label{alg:factorial}
\end{algorithm}

\subsubsection{Punktas}
\subsubsubsection{Papunktis}
\subsubsection{Punktas}
\section{Skyrius}
\subsection{Poskyris}
\subsection{Poskyris}

\sectionnonum{Rezultatai ir išvados}
Rezultatų ir išvadų dalyje išdėstomi pagrindiniai darbo rezultatai (kažkas
išanalizuota, kažkas sukurta, kažkas įdiegta), toliau pateikiamos išvados
(daromi nagrinėtų problemų sprendimo metodų palyginimai, siūlomos
rekomendacijos, akcentuojamos naujovės). Rezultatai ir išvados pateikiami
sunumeruotų (gali būti hierarchiniai) sąrašų pavidalu. Darbo rezultatai turi
atitikti darbo tikslą.

\printbibliography[heading=bibintoc]  % Šaltinių sąraše nurodoma panaudota
% literatūra, kitokie šaltiniai. Abėcėlės tvarka išdėstomi darbe panaudotų
% (cituotų, perfrazuotų ar bent paminėtų) mokslo leidinių, kitokių publikacijų
% bibliografiniai aprašai. Šaltinių sąrašas spausdinamas iš naujo puslapio.
% Aprašai pateikiami netransliteruoti. Šaltinių sąraše negali būti tokių
% šaltinių, kurie nebuvo paminėti tekste (LaTeX tai sutvarko automatiškai).
% Šaltinių sąraše rekomenduojame necituoti savo kursinio darbo, nes tai nėra
% oficialus literatūros šaltinis. Jei tokių nuorodų reikia, pateikti jas tekste.

% \sectionnonum{Sąvokų apibrėžimai}
\sectionnonum{Santrumpos}
Sąvokų apibrėžimai ir santrumpų sąrašas sudaromas tada, kai darbo tekste
vartojami specialūs paaiškinimo reikalaujantys terminai ir rečiau sutinkamos
santrumpos.

% Priedai
% Prieduose gali būti pateikiama pagalbinė, ypač darbo autoriaus savarankiškai
% parengta, medžiaga. Savarankiški priedai gali būti pateikiami ir
% kompaktiniame diske. Priedai taip pat numeruojami ir vadinami. Darbo tekstas
% su priedais susiejamas nuorodomis.
\appendix{Neuroninio tinklo struktūra}

\begin{figure}[H]
    \centering
    \includegraphics[scale=0.5]{img/MLP}
    \caption{Paveikslėlio pavyzdys}
    \label{img:mlp}
\end{figure}


\appendix{Eksperimentinio palyginimo rezultatai}

% tablesgenerator.com - converts calculators (e.g. excel) tables to LaTeX
\begin{table}[H]\footnotesize
  \centering
  \caption{Lentelės pavyzdys}
  {\begin{tabular}{|l|c|c|} \hline
    Algoritmas & $\bar{x}$ & $\sigma^{2}$ \\
    \hline
    Algoritmas A  & 1.6335    & 0.5584       \\
    Algoritmas B  & 1.7395    & 0.5647       \\
    \hline
  \end{tabular}}
  \label{tab:table example}
\end{table}

\end{document}
